\documentclass[10pt,fleqn]{article} % Default font size and left-justified equations
\usepackage[%
    pdftitle={Modélisation dynamique},
    pdfauthor={Xavier Pessoles}]{hyperref}

    
\input{style/new_style}
\input{style/macros_SII}
\usepackage{multicol}
\usepackage{siunitx}
%\usepackage{picins}
\fichetrue
%\fichefalse

\proftrue
\proffalse

\tdtrue
%\tdfalse

\courstrue
\coursfalse


\def\classe{\textsf{PSI$\star$ -- MP}}
\def\xxnumpartie{Cycle 07}
\def\xxpartie{Modélisation des chaînes de solides dans le but de déterminer les contraintes géométriques dans les mécanismes}

\def\xxnumchapitre{Chapitre 2 \vspace{.2cm}}
\def\xxchapitre{\hspace{.12cm} Hyperstatisme}


\def\discipline{Sciences \\Industrielles de \\ l'Ingénieur}
\def\xxtete{Sciences Industrielles de l'Ingénieur}




\def\xxtitreexo{Application 02}
\def\xxsourceexo{\hspace{.2cm} \footnotesize{Pôle Chateaubriand -- Joliot-Curie}}


\def\xxposongletx{2}
\def\xxposonglettext{1.45}
\def\xxposonglety{20}
%\def\xxonglet{Part. 1 -- Ch. 3}
\def\xxonglet{\textsf{Cycle 07}}

\def\xxactivite{Application}
\def\xxauteur{\textsl{Pôle Chateaubriand -- Joliot-Curie}}

\def\xxcompetences{%
\vspace{-.5cm}
\footnotesize{
\textsl{%
\textbf{Savoirs et compétences :}\\
\vspace{-.2cm}
\begin{itemize}[label=\ding{112},font=\color{ocre}] 
\item \textit{Mod2.C34} : chaînes de solides;
\item \textit{Mod2.C34} : degré de mobilité du modèle;
\item \textit{Mod2.C34} : degré d’hyperstatisme du modèle;
\item \textit{Mod2.C34.SF1} : déterminer les conditions géométriques associées à l’hyperstatisme;
\item \textit{Mod2.C34} : résoudre le système associé à la fermeture cinématique et en déduire le degré de mobilité et d’hyperstatisme.
\end{itemize}}}}

\def\xxfigures{
%\includegraphics[width=.6\linewidth]{images/fig_00}
}%figues de la page de garde

\def\xxpied{%
Cycle 07 -- Modélisation des chaînes de solides \\%dans le but de déterminer les contraintes géométriques dans les mécanismes\\% afin de valider leurs performances.\\
Chapitre 2 -- \xxactivite%
}

\setcounter{secnumdepth}{5}
%---------------------------------------------------------------------------

\usepackage{pgfplots}
\begin{document}
\def\pathfig{images}
%\chapterimage{png/Fond_Cin}
\input{style/new_pagegarde}
\vspace{4.5cm}
\pagestyle{fancy}
\thispagestyle{plain}

\def\columnseprulecolor{\color{ocre}}
\setlength{\columnseprule}{0.4pt} 

\def\pathfig{images}

\begin{multicols}{2}



\section*{Système freinage du TGV}
\subsection*{Présentation}
Une rame de TGV est en général composée de deux motrices
et de huit voitures.
La liaison avec les rails est assurée par bogies. Quatre
d’entre eux, implantés sous les motrices, sont moteurs, les
neuf autres, qualifiés de porteurs, sont positionnés entre
deux voitures.

\begin{center}
\includegraphics[width=\linewidth]{images/fig_01.png}
\end{center}


\begin{center}
\includegraphics[width=\linewidth]{images/fig_02.png}
\end{center}

\begin{center}
\includegraphics[width=\linewidth]{images/fig_03.png}
\end{center}

Un bogie porteur est un chariot à deux essieux et quatre roues.
Il supporte en sa partie supérieure l’une des extrémités de la
voiture et permet de suivre les courbes de la voie.
Chacune des roues est équipée d’un système de freinage à
disques et contribue à l’arrêt de la voiture.
Le système de freinage qui équipe un bogie porteur est détaillé
ci-dessous.

\begin{center}
\includegraphics[width=\linewidth]{images/fig_04.png}
\end{center}


\begin{center}
\includegraphics[width=\linewidth]{images/fig_05.png}
\end{center}

Le modèle cinématique de ce système de freinage est alors le suivant :

\begin{center}
\includegraphics[width=\linewidth]{images/fig_06.png}
\end{center}


\begin{center}
\includegraphics[width=\linewidth]{images/fig_07.png}
\end{center}

\begin{obj}
Analyser la conception du système de freinage du TGV.
\end{obj}

\begin{enumerate}
\item Réaliser le graphe des liaisons du système de freinage sans le disque.
\item Déterminer le degré d’hyperstatisme du mécanisme lorsque les garnitures ne sont pas en contact
avec le disque.
\item Justifier la nécessité d’avoir un mécanisme hyperstatique dans ce cas.
\item Indiquer, d’un point de vue effort et sans calcul, l’utilité des biellettes.
\item Il existe sur le TGV d’autres dispositifs de freinage. Indiquer, en précisant le phénomène physique qui intervient, au moins deux autres principes de dissipation de l’énergie pouvant être utilisés.
\end{enumerate}

%\vspace{3cm}
%
%\subsection*{Éléments de corrigé}
\end{multicols}

\end{document}



\begin{center}
\includegraphics[width=.8\linewidth]{images/fig_01.png}
\end{center}


\subparagraph{}
\textit{}
\ifprof
\begin{corrige}
\end{corrige}\else\fi


