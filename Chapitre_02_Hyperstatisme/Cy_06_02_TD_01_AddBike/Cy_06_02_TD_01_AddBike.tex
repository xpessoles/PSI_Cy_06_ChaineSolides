% "{'classe':('PSI'),'chapitre':'chs_hs','type':('td'),'titre':'Suspension de l\\'AddBike', 'source':'Agrégation Sciences Industrielles de l\\'Ingénieur - 2018','comp':['B2-16',],'corrige':False}"
%\setchapterimage{bandeau}
\chapter*{TD \arabic{cptTD} :\\ 
Suspension de l'AddBike -- \ifprof Corrigé \else Sujet \fi}
\addcontentsline{toc}{section}{TD \arabic{cptTD} : Suspension de l'AddBike -- \ifprof Corrigé \else Sujet \fi}

\iflivret \stepcounter{cptTD} \else
\ifprof  \stepcounter{cptTD} \else \fi
\fi

\setcounter{question}{0}
\marginnote{Agrégation Sciences Industrielles de l'Ingénieur -- 2018.}
\marginnote{
\xpComp{CHS}{01}
\xpComp{CHS}{02}
%\UPSTIcompetence[2]{B2-16}
}

\begin{marginfigure}
\includegraphics[width=\linewidth]{add_03}
\end{marginfigure}


\subsection*{Présentation}
L'Add-Bike est un système pouvant s’adapter à tous types de vélo et doit permettre de transporter des marchandises (colis ou courses du quotidien) ou des enfants.
\begin{marginfigure}
\includegraphics[width=\linewidth]{add_02.png}
\end{marginfigure}
Il est équipé d'un système de suspension permettant de limiter le mouvement de roulis dans les virages. 

\begin{center}
\includegraphics[width=.7\linewidth]{add_04.png}
\end{center}

\subsection*{Exigence 1.2 : Stabilité des occupants et des marchandises}

\begin{obj}
Pour assurer la stabilité des occupants du bi-roue, il est nécessaire de déterminer les conditions géométriques permettant de limiter l’angle de roulis (exigence 1.2.1). Ainsi, cet angle roulis ne doit pas dépasser $\beta=5\degres$  lorsque le cycliste  penche le mât vertical de $\alpha=30\degres$.
%Ces conditions géométriques devront être respectées lors de la conception de la fusée (exigence 1.2.2).
\end{obj}

\begin{center}
\includegraphics[width=\linewidth]{req_01.png}
\end{center}
Pour pouvoir tourner, le cycliste penche le mât vertical 04 par l’intermédiaire du guidon, ce qui conduit à la déformation du parallélogramme $ACDF$ donné dans la figure suivante et à la rotation des roues autour de l’axe horizontal longitudinal
$\vect{x_0}$. Lors de la déformation du parallélogramme, les bielles 01 et 03 ne restent pas parfaitement horizontales ; le passager assis dans le siège lié à la bielle 03, subit donc du roulis, c’est-à-dire un pivotement autour de l’axe horizontal longitudinal $\vect{x_0}$.

\begin{center}
\includegraphics[width=.7\linewidth]{add_05.png}
\end{center}

L’angle $\beta$  correspond à l’angle de roulis des bielles 01 et 03.

\question{En réalisant une fermeture géométrique, déterminer la relation liant l’angle $\beta$  et l’excentricité $e$ des fusées 02g et 02d.}

\question{En déduire une valeur de l’excentricité $e$ permettant de valider l’exigence 1.2.1.}

\ifprof
\begin{corrige}
\end{corrige}\else\fi


\subsection*{Exigence 1.5 : Exigences économiques -- Assemblage}
\begin{obj}
Afin de pouvoir vendre son produit à un prix attractif, la start-up doit pouvoir fabriquer et assembler son produit à un coût satisfaisant. Une maîtrise des coûts passe par la maîtrise des spécifications garantissant l’assemblage du système et par des coûts de fabrication réduits. Les objectifs sont ici de : spécifier des conditions géométriques sur les dimensions de la bielle inférieure (03) à partir des conditions de fonctionnement.
\end{obj}

\begin{center}
\includegraphics[width=\linewidth]{req_02.png}
\end{center}

\question{Après avoir fait un graphe de structure et sans tenir compte des roues et de leurs liaisons au sol, donner le degré d’hyperstatisme du modèle cinématique suivant.}

\ifprof
\begin{corrige}
\end{corrige}\else\fi

\begin{center}
\includegraphics[width=.7\linewidth]{add_01.png}
\end{center}


\question{Donner les torseurs cinématiques $\torseurcin{V}{2}{3}$, $\torseurcin{V}{1}{2}$, $\torseurcin{V}{4}{3}$, $\torseurcin{V}{1}{4}$.}
\ifprof
\begin{corrige}
\end{corrige}\else\fi


\question{En utilisant une fermeture de chaîne cinématique, donner le système d’équations liant les différentes variables.}
\ifprof
\begin{corrige}
\end{corrige}\else\fi



\question{En déduire les conditions géométriques à imposer sur la bielle (03) afin de satisfaire l’assemblage du mécanisme. }
\ifprof
\begin{corrige}
\end{corrige}\else\fi


\ifprof
\else
\begin{marginfigure}
\centering
\includegraphics[width=3cm]{Cy_06_02_TD_01_AddBike_qr}
\end{marginfigure}
\fi


\subsection*{Synthèse}
\question{Conclure sur les méthodes qui ont permis de répondre aux exigences 1.4 et 1.5.}
\ifprof
\begin{corrige}
\end{corrige}\else\fi

